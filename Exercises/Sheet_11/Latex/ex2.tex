\begin{itemize}  
    \item[\textbf{(a)}]
    \begin{proof}
We will proof that $k(x,x^{\prime})$ is positive semi-definit. We will do so by prooving that :
\[
\forall x_1,\cdots ,x_n \in \R \text{ and }\forall c_1,\cdots ,c_n \in \R   : \sum_{i=1}^n\sum_{j=1}^n c_ic_jk(x_i,x_j) \geq 0
\]
We have: 
\begin{align*}
    \sum_{i=1}^n\sum_{j=1}^n c_ic_jk(x_i,x_j)   &= \sum_{i=1}^n\sum_{j=1}^n c_ic_j(30x_ix_j+1)\\
                                                &= \sum_{i=1}^n\sum_{j=1}^n(30x_ix_j c_ic_j+ c_ic_j)\\
                                                &= 30\sum_{i=1}^n\sum_{j=1}^nx_ix_j c_ic_j + \sum_{i=1}^n\sum_{j=1}^nc_ic_j \\
                                                &=30(\sum_{i=1}^nc_ix_i)^T\sum_{j=1}^nc_jx_j + (\sum_{i=1}^nc_i)^T\sum_{j=1}^nc_j \\
                                                &=30||\sum_{i=1}^nx_ic_i||_2^2+||\sum_{i=1}^nc_i||_2^2 \geq 0
\end{align*}
\end{proof}
\item[\textbf{(b)}] With the help of the observed Data and the given Kernel function we can calculate the Covariance Matrix:
\[
k_{XX}= \begin{pmatrix}
    31 & 29\\
    29 & 31
    \end{pmatrix}    
\]
Additionally according to the lecture $\Lambda$ is give by:
\[
  \Lambda= \sigma^2I =   I
\]
We can now use the expressions from the lecture to calculate $\mu_*$ and $sigma^2_*$
\begin{align*}
    \mu_*   &= m_{X_*}+k_{X_*X}(k_{XX}+\Lambda)^{-1}(y-m_X) \\
            &= \begin{pmatrix}
                31 & 29
                \end{pmatrix} 
                \begin{pmatrix}
                32 & 29\\
                29 & 32
                \end{pmatrix}^{-1}
                \begin{pmatrix}
                    1 \\
                    1 
                \end{pmatrix} \\
            &= 0.984
\end{align*}
\begin{align*}
    \sigma^2_*      &= m_{X_*}+k_{X_*X}(k_{XX}+\Lambda)^{-1}(y-m_X) \\
                    &= k_{X_*X_*}-k_{X_*X}(k_{XX}+\Lambda)^{-1}k_{XX_*}+\sigma^2\\
                    &=31 - \begin{pmatrix}
                        31 & 29
                        \end{pmatrix} 
                        \begin{pmatrix}
                        32 & 29\\
                        29 & 32
                        \end{pmatrix}^{-1}\begin{pmatrix}
                            29 \\
                            32 
                        \end{pmatrix} +1 \\
                    &= 4
\end{align*}
\end{itemize}