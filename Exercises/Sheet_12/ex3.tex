\begin{itemize}
    \item[\textbf{(a)}] 
    \begin{proof} 
        Let $k(x_1,x_2)= C$ with $C \in \R_{>0}$.  Then for $x \in \R^n$ we have:
    \[
    x^T k_{\textbf{xx}}x = C(\sum_{i=1}^nx_i)(\sum_{j=1}^nx_j)    
    \]
    We will show that this sum is greater or equal to 0. To show that let I be the set of indices from 1 to n. Let further be :
    \[
    P \subseteq I := \{i\in I : x_i \geq 0 \}   
    \]
    \[
    N \subseteq I := \{i\in I : x_i < 0 \}   
    \]
    Then we can write :
    \[
        C(\sum_{i=1}x_i)(\sum_{j=1}x_j)  =   C(\sum_{i \in P}x_i+ \sum_{j \in N}x_j)(\sum_{l \in P}x_l+ \sum_{k \in N}x_k)  
    \]
    We can now distinguish two cases:

    \textbf{Case 1: $\sum_{i\in P} x_i \geq \sum_{j \in N} |x_j|$}
    Then we have 
    \[
        C(\underbrace{\sum_{i \in P}x_i+ \sum_{j \in N}x_j}_{\geq 0})(\underbrace{\sum_{l \in P}x_l+ \sum_{k \in N}x_k}_{\geq 0}) \geq 0    
    \]
    \textbf{Case 2: $\sum_{i\in P} x_i < \sum_{j \in N} |x_j|$}
    Then we have 
    \[
        C \underbrace{(\underbrace{\sum_{i \in P}x_i+ \sum_{j \in N}x_j}_{< 0})(\underbrace{\sum_{l \in P}x_l+ \sum_{k \in N}x_k}_{< 0})}_{>0} > 0    
    \]
    So we have 
    \[
        x^T k_{\textbf{xx}}x = C(\sum_{i=1}^nx_i)(\sum_{j=1}^nx_j) \geq 0
    \]
    And k is thus positive semidefinite.
    \end{proof}
    
     \item[\textbf{(b)}] 
    		\begin{proof} 
    			Let $k(x_1,x_2)= x_1 \cdot x_2$ with $X = \R$. 
    			
    			It follows:
    				\begin{align*}
    					\sum_{i=1}^n\sum_{j=1}^n c_i \cdot c_j \cdot k(x_i, x_j) 
    					&=\sum_{i=1}^n\sum_{j=1}^n c_i \cdot c_j \cdot x_i \cdot x_j\\
    					&=\sum_{i=1}^n c_i \cdot x_i \sum_{i=1}^n c_i\cdot x_i 
    					&=(\sum_{i=1}^n c_i \cdot x_i)^2 \geq 0
    				\end{align*}
    				Thus k is positive semidefinite.
     	\end{proof}
     	
     	\item[\textbf{(c)}] 
    		\begin{proof}
    			Let $k(x_1,x_2)= x_1 + x_2$ with $X = \R$. 
    			We have x $\in \R$.  So with x=-1 it is:
    			\[(-1) \cdot k(-1,-1) \cdot (-1) = (-1) \cdot (-2) \cdot (-1) = -2 < 0\]
    			Therefore $k(x_1,x_2)= x_1 + x_2$ is not positive semidefinite and thus is not a kernel.
    		\end{proof}
    		
    		\item[\textbf{(d)}] 
    		\begin{proof}
    			Let $k(x_1,x_2)= 5 \cdot x_1^T \cdot x_2$ with $X = \R^D$. 
    			
    			It follows:
    				\begin{align*}
    					\sum_{i=1}^n\sum_{j=1}^n c_i \cdot c_j \cdot k(x_i, x_j)
    					&=\sum_{i=1}^n\sum_{j=1}^n c_i \cdot c_j \cdot 5 \cdot x_i^T \cdot x_j  \\
    					&=5\cdot (\sum_{i=1}^n c_i \cdot x_i)^T  \cdot (\sum_{i=1}^n c_i \cdot x_i) 
    					&=5\cdot \|\sum_{i=1}^n c_i \cdot x_i\|_2^2 \geq 0
    				\end{align*}
    				Thus k is positive semidefinite.
    		\end{proof}
    		
    		\item[\textbf{(e)}] 
    		\begin{proof}
    			Let $k(x_1,x_2)= (x_1^T \cdot x_2 + 1)^2$ with $X = \R^N$. 
    			\begin{align*}
    				\sum_{i=1}^n\sum_{j=1}^n c_i \cdot c_j \cdot k(x_i,x_j)   
    				&= \sum_{i=1}^n\sum_{j=1}^n c_i \cdot c_j \cdot (x_i \cdot x_j+1) \\
                  &= \sum_{i=1}^n\sum_{j=1}^n c_i \cdot c_j \cdot x_i \cdot x_j + c_i \cdot c_j\\
                  &= \sum_{i=1}^n\sum_{j=1}^nx_i \cdot x_j \cdot c_i \cdot c_j + \sum_{i=1}^n\sum_{j=1}^nc_ i \cdot c_j \\
                  &=(\sum_{i=1}^nc_i \cdot x_i)^T\sum_{j=1}^nc_j \cdot x_j + (\sum_{i=1}^nc_i)^T\sum_{j=1}^nc_j \\
                  &=||\sum_{i=1}^nc_i \cdot x_i||_2^2+||\sum_{i=1}^nc_i||_2^2 \geq 0
			\end{align*}
			Thus k is positive semidefinite.

		\end{proof}
\end{itemize}