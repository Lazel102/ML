%This is my super simple Real Analysis Homework template

\documentclass{article}
\usepackage[ngerman]{babel}
\usepackage[]{amsthm} %lets us use \begin{proof}
\usepackage[]{amssymb} %gives us the character \varnothing
\usepackage{mathtools}
\usepackage{lineno}
\usepackage[ansinew]{inputenc}
\usepackage{graphicx}
\newcommand{\dsep}{\rotatebox[origin=c]{90}{$\models$}}
\newcommand{\R}{\mathbb{R}}
\newcommand{\uproman}[1]
{\uppercase\expandafter{\romannumeral#1}}
\title{Machine Learning Exercise Sheet 2}
\author{Yannick Zelle and Nina Fischer}
\date\today
%This information doesn't actually show up on your document unless you use the maketitle command below

\begin{document}

\maketitle %This command prints the title based on information entered above
\section*{Exercise 1}
\subsection*{(a)}
Given is the DAG $G_1$ with the following associated probabilities:
\begin{itemize}
    \item $p(A)=0.3$
    \item $p(B \mid A)=0.2$
    \item $p(B \mid \neg A)=0.4$
    \item $p(C \mid B)=0.7$
    \item $p(C \mid \neg B)=0.5$
\end{itemize}

From there we can deduct the following probabilities:

\begin{itemize}
    \item $p(\neg A)=0.7$
    \item $p(\neg B \mid A)=0.8$
    \item $p(\neg B \mid \neg A)=0.6$
    \item $p(\neg C \mid B)=0.3$
    \item $p(\neg C \mid \neg B)=0.5$
\end{itemize}

From $G_1$ we can deduct the joint probability distribution:

\begin{align*}
    p(A,B,C)=p(A) \cdot p(B \mid A) \cdot p(C \mid B)
\end{align*}

Using this Probabilitydistribution we can calculate P(B):

\begin{align*}
    p(B)    &= \sum_{A,C}^{} p(A,B=1,C) \\
            &= p(A) \cdot p(B \mid A) \cdot p(C \mid B) \\
            &+ p(\neg A) \cdot p(B \mid \neg A) \cdot p(C \mid B) \\
            &+ p(\neg A) \cdot p(B \mid \neg A) \cdot p(\neg C \mid B) \\
            &+ p( A) \cdot p(B \mid  A) \cdot p(\neg C \mid B) \\
            &= 0.3 \cdot 0.2 \cdot 0.7 \\
            &+ 0.7 \cdot 0.4 \cdot 0.7 \\
            &+ 0.7 \cdot 0.4 \cdot 0.3 \\
            &+ 0.3\cdot 0.2 \cdot 0.3 \\
            &= 0.34
\end{align*}

and P(C):

\begin{align*}
    p(C)    &= \sum_{A,B}^{} p(A,B,C=1) \\
            &= p(A) \cdot p(B \mid A) \cdot p(C \mid B) \\
            &+ p(\neg A) \cdot p(B \mid \neg A) \cdot p(C \mid B) \\
            &+ p(\neg A) \cdot p(\neg B \mid \neg A) \cdot p(C \mid \neg B) \\
            &+ p( A) \cdot p(\neg B \mid  A) \cdot p( C \mid \neg B) \\
            &= 0.3 \cdot 0.2 \cdot 0.7 \\
            &+ 0.7 \cdot 0.4 \cdot 0.7 \\
            &+ 0.7 \cdot 0.6 \cdot 0.5 \\
            &+ 0.3 \cdot 0.8 \cdot 0.5 \\
            &= 0.568
\end{align*}

\subsection*{(b)}
Given is a DAG $G_2$ and the associated probabilities:
\begin{itemize}
    \item $p(A)=0.3$
    \item $p(B \mid A)= 0.2$
    \item $p(B \mid \neg A) = 0.4$
    \item $p(c \mid A)= 0.7$
    \item $p(C \mid \neg A)= 0.6$
    \item $p(D \mid B,C)=0.9$
    \item $p(D \mid B, \neg C)=0.5$
    \item $p(D \mid \neg B, C)=0.3$
    \item $p(D \mid \neg B, \neg C)= 0.3$
\end{itemize}

From there we can deductthe following probabilities:
\begin{itemize}
    \item $p(\neg A)=0.7$
    \item $p(\neg B \mid A)= 0.8$
    \item $p(\neg B \mid \neg A) = 0.6$
    \item $p(\neg c \mid A)= 0.3$
    \item $p(\neg C \mid \neg A)= 0.4$
    \item $p(\neg D \mid B,C)=0.1$
    \item $p(\neg D \mid B, \neg C)=0.5$
    \item $p(\neg D \mid \neg B, C)=0.7$
    \item $p(\neg D \mid \neg B, \neg C)= 0.7$
\end{itemize}

From the $G_2$ we can deduct the following joint Probabilitydistribution:

\[
p(A,B,C,D)= p(A) \cdot p(C \mid A) \cdot p(B \mid A) \cdot p(D \mid B,C)    
\]

Using the joint Probabilitydistribution and the probabilities from above, we can calculate $p(B)$:

\begin{align*}
p(B)    &= \sum_{A,C,D} p(A,B=1,C, D) \\
        &= p(A) \cdot p(C \mid A) \cdot p(B \mid A) \cdot p(D \mid B,C) \\
        &+ p(A) \cdot p(C \mid A) \cdot p(B \mid A) \cdot p(\neg D \mid B,C) \\
        &+ p(A) \cdot p(\neg C \mid A) \cdot p(B \mid A) \cdot p(D \mid B,\neg C) \\
        &+ p(A) \cdot p(\neg C \mid A) \cdot p(B \mid A) \cdot p(\neg D \mid B,\neg C) \\
        &+ p(\neg A) \cdot p(C \mid \neg A) \cdot p(B \mid \neg A) \cdot p(D \mid B,C) \\
        &+ p(\neg A) \cdot p(C \mid \neg A) \cdot p(B \mid \neg A) \cdot p(\neg D \mid B,C) \\
        &+ p(\neg A) \cdot p(\neg C \mid \neg A) \cdot p(B \mid \neg A) \cdot p(D \mid B,\neg C) \\
        &+ p(\neg A) \cdot p(\neg C \mid \neg A) \cdot p(B \mid \neg A) \cdot p(\neg D \mid B,\neg C) \\
        &= 0.7 \cdot 0.7 \cdot 0.2 \cdot 0.9\\
        &+ 0.7 \cdot 0.3 \cdot 0.8 \cdot 0.1 \\
        &+ 0.7 \cdot 0.3\cdot 0.2\cdot 0.5 \\
        &+ 0.7 \cdot 0.3 \cdot 0.2 \cdot 0.5 \\
        &+ 0.3 \cdot 0.6 \cdot 0.4 \cdot 0.9 \\
        &+ 0.3\cdot 0.6 \cdot 0.4\cdot 0.1 \\
        &+ 0.3 \cdot 0.4 \cdot 0.4 \cdot 0.5 \\
        &+ 0.3\cdot 0.4 \cdot 0.4 \cdot 0.5 \\
        &= 0.27
\end{align*}

and p(D):

\begin{align*}
p(D)    &= \sum_{A,C,B} p(A,B,C, D=1)\\
        &= p(A) \cdot p(C \mid A) \cdot p(B \mid A) \cdot p(D \mid B,C)\\
        &+ p(A) \cdot p(C \mid A) \cdot p(\neg B \mid A) \cdot p( D \mid \neg B,C) \\
        &+ p(A) \cdot p(\neg C \mid A) \cdot p(B \mid A) \cdot p(D \mid B,\neg C) \\
        &+ p(A) \cdot p(\neg C \mid A) \cdot p(\neg B \mid A) \cdot p( D \mid \neg B,\neg C) \\
        &+ p(\neg A) \cdot p(C \mid \neg A) \cdot p(B \mid \neg A) \cdot p(D \mid B,C) \\
        &+ p(\neg A) \cdot p(C \mid \neg A) \cdot p\neg (B \mid \neg A) \cdot p(D \mid \neg B,C) \\
        &+ p(\neg A) \cdot p(\neg C \mid \neg A) \cdot p(B \mid \neg A) \cdot p(D \mid B,\neg C) \\
        &+ p(\neg A) \cdot p(\neg C \mid \neg A) \cdot p(\neg B \mid \neg A) \cdot p(D \mid \neg B,\neg C) \\
        &= 0.7 \cdot 0.7 \cdot 0.2 \cdot 0.9 \\
        &+ 0.7 \cdot 0.3 \cdot 0.8 \cdot 0.3 \\
        &+ 0.7 \cdot 0.3\cdot 0.2 \cdot 0.5 \\
        &+ 0.7 \cdot 0.3 \cdot 0.8 \cdot 0.3 \\
        &+ 0.3 \cdot 0.6 \cdot 0.4 \cdot 0.9 \\
        &+ 0.3 \cdot 0.6 \cdot 0.6\cdot 0.3 \\
        &+ 0.3 \cdot 0.4 \cdot 0.4 \cdot 0.5 \\
        &+ 0.3 \cdot 0.4 \cdot 0.6 \cdot 0.3 \\
        &= 0.36
\end{align*}



\section*{Exercise 2}
\subsection*{(a)}
\[
  p(A, B, C, D, E, F ) = p(A) \cdot p(B \mid A) \cdot p(C \mid B) \cdot p(D \mid B) \cdot p(E \mid C,D) \cdot p(F \mid E)
\]

\subsection*{(b)}
\[
p(A, B, C, D, E ) = p(A) \cdot p(B \mid A) \cdot p(C \mid B) \cdot p(D \mid B) \cdot p(E \mid C,D) 
\]

\subsection*{(c)}
\textbf{Task:} Find a minimal set S that d-seperates A and F and prove that this is the case.
\begin{proof}
    We propose $S= \{ B \}$ to prove that the criteria of S holds we have actually to prove two statements:
    \begin{enumerate}
        \item S d-seperates A and F
        \item there is no set with fewer elements that also d-seperates A and F
    \end{enumerate}
     
We will start with the first statement. Between A and F exist two paths:
\begin{align*}
    p_1 = A \rightarrow B \rightarrow C \rightarrow E \rightarrow F \\
    p_2 = A \rightarrow B \rightarrow D \rightarrow E \rightarrow F 
\end{align*}

We have $p_1$ is blocked by S because with $i_k = B$ we have : $i_k \in S$ and $A \rightarrow B \rightarrow C \leftrightarrow i_{k-1} \rightarrow i_k \rightarrow i_{k+1}$ and $p_2$ is blocked by S because with $i_k = B$ we have : $i_k \in S$ and $A \rightarrow B \rightarrow D \leftrightarrow i_{k-1} \rightarrow i_k \rightarrow i_{k+1}$
It is left to show that there is no set with fewer elements that also d-seperates A and F. The only set that has fewer elements is the empty set but the empty set is neither blocking $p_1$ nor $p_2$ according to the definition.
\end{proof}

\subsection*{(d)}
\begin{proof}
    
    We will proof that $C \dsep_G D \mid B$ i.e B d-seperates C and D holds. 
    This is the case if every path is blocked by $S=\{B\}$. There are two paths from C to D:
    \begin{align*}
        p_1 = C \leftarrow B \rightarrow D \\
        p_2 = C \rightarrow E \leftarrow D
    \end{align*}
    $p_1$ is blocked by S because with $i_k=B$ we have $i_k \in S$ and $C \leftarrow B \rightarrow D \leftrightarrow i_{k-1} \leftrightarrow \leftarrow i_k \rightarrow i_{k+1}$.
    Also $p_2$ is blocked by $S$ because with $i_k=E$ we have : $i_k \notin S$ and $C \rightarrow E \leftarrow D \leftrightarrow i_{k-1} \rightarrow i_k \leftarrow i_{k+1}$
    So every path between C and D is blocked by S and therefore $C \dsep_G D \mid B$ holds.
\end{proof}

\section*{Exercise 3}
\subsection*{(a)}
We will proof that 
\begin{align*}
    A \rightarrow B \rightarrow C \implies A \dsep C \mid B
\end{align*}
\begin{proof}
\begin{align*}
	p(A,B,C) = p(A) \cdot p(B \mid A) \cdot p(C\mid B) \\
	\Leftrightarrow p(A,B,C) = p(A, B) \cdot p(C\mid B) \\
	\Leftrightarrow p(A,B,C) = p(A\mid B) \cdot p(B) \cdot p(C\mid B) \\
	\Leftrightarrow \frac{p(A,B,C)}{p(B)} = p(A\mid B) \cdot p(C\mid B) \\
	\Leftrightarrow p(A,C \mid B) = p(A, C \mid B)
\end{align*}
\end{proof}

\subsection*{(b)}
We will proof that 
\begin{align*}
    A \leftarrow B \leftarrow C \implies A \dsep C \mid B
\end{align*}
and we know from a that $p(C) \cdot p(B \mid C) \Leftrightarrow p(B) \cdot p(C \mid B)$.
\begin{proof}
\begin{align*}
	p(A,B,C) = p(A \mid B) \cdot p(B \mid C) \cdot p(C) \\
	\Leftrightarrow p(A,B,C) = p(A\mid B) \cdot p(C\mid B) \cdot p(B) \\
	\Leftrightarrow \frac{p(A,B,C)}{p(B)} = p(A\mid B) \cdot p(C\mid B) \\
	\Leftrightarrow p(A,C \mid B) = p(A, C \mid B)
\end{align*}
\end{proof}

\subsection*{(c)}
We will proof that 
\begin{align*}
    A \leftarrow B \rightarrow C \implies A \dsep C \mid B
\end{align*}
\begin{proof}
\begin{align*}
	p(A,B,C) = p(A \mid B) \cdot p(B) \cdot p(C \mid B)  \\
	\Leftrightarrow \frac{p(A,B,C)}{p(B)} = p(A\mid B) \cdot p(C\mid B) \\
	\Leftrightarrow p(A,C \mid B) = p(A, C \mid B)
\end{align*}
\end{proof}

\subsection*{(d)}
We will proof that 
\begin{align*}
    A \rightarrow B \leftarrow C \implies A \dsep C 
\end{align*}
\begin{proof}
\begin{align*}
	p(A,B,C) = p(A) \cdot p(B \mid A) \cdot p(B \mid C) \cdot p(C) \\
	\Leftrightarrow \frac{p(A,B,C)}{p(A) \cdot p(C)} = p(B\mid A) \cdot p(B\mid C) \\
	\Leftrightarrow p(B \mid A,C) = p(B \mid A,C)
\end{align*}
\end{proof}


%Section and subsection automatically number unless you put the asterisk next to them.


%Basically, you type whatever text you want and use the $ sign to enter "math mode".
%For fancy calligraphy letters, use \mathcal{}
%Special characters are their own commands





\end{document}
