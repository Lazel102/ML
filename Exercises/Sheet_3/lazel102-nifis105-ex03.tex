%This is my super simple Real Analysis Homework template

\documentclass{article}
\usepackage[english]{babel}
\usepackage[]{amsthm} %lets us use \begin{proof}
\usepackage[]{amssymb} %gives us the character \varnothing
\usepackage{mathtools}
\usepackage{lineno}
\usepackage[ansinew]{inputenc}
\newcommand{\R}{\mathbb{R}}
\newcommand{\uproman}[1]{\uppercase\expandafter{\romannumeral#1}}
\title{Solutions Sheet 3}
\author{Nina Fischer and Yannick Zelle}
\date\today
%This information doesn't actually show up on your document unless you use the maketitle command below

\begin{document}
\maketitle %This command prints the title based on information entered above

%Section and subsection automatically number unless you put the asterisk next to them.


%Basically, you type whatever text you want and use the $ sign to enter "math mode".
%For fancy calligraphy letters, use \mathcal{}
%Special characters are their own commands

\section*{Exercise 1}
 \textbf{Task:} Describe the maximum likelihood estimator for the following distributions:

 \subsection*{(a)}
 \[
 \mathcal{N}(x \mid \mu, \sigma^2)
 \]

 We denote with $f_\mu$ the PDF of the normal distribution. By using the Lemma 6.2.3 we can write the log-likelihood function as:

 \begin{align*}
     l(\mu) &= ln(\prod^n_{i=1} f_{\mu})\\
            &=ln(\prod^n_{i=1} e^{a+\eta x_i -\frac{1}{2}\lambda^2x_i^2})\\
            &= \sum^n_{i=1}ln(e^{a+\eta x_i -\frac{1}{2}\lambda^2x_i^2})\\
            &= \sum^n_{i=1} -\frac{1}{2}(log(2\pi)-log(\lambda^2)+\frac{\mu^2}{\sigma^2}) + \frac{\mu}{\sigma^2}x_i-\frac{1}{2}\lambda^2x_i^2\\
 \end{align*}

To find $\mu$ which maximizes $l(\mu)$ we calculate:

\begin{align*}
    \frac{\delta l(\mu)}{\delta \mu}    &= \sum_{i=1}^n -\frac{\mu}{\sigma^2} + \frac{1}{\sigma^2}x_i  \\
                                        &= -n\frac{\mu}{\sigma^2} + \frac{1}{\sigma^2} \sum_{i=1}^n x_i
\end{align*}
 $l(\mu)$ is maximal if :

 \begin{align*}
     l(\mu) = 0 & \leftrightarrow 0 = -n\frac{\mu}{\sigma^2} + \frac{1}{\sigma^2} \sum_{i=1}^n x_i\\
                & \leftrightarrow \mu =  \frac{1}{n} \sum_{i=1}^n x_i
    \end{align*}

Since $\frac{\delta^2 l(\mu)}{\delta^2 \mu} = -\frac{1}{\sigma^2}< 0$, with $\mu_o := \frac{1}{n} \sum_{i=1}^n x_i$ we obtain a global maximum for $l(\mu)$ and we chose $\mu_0$ therefore as our estimator.

\subsection*{(b)}

Exponential distribution with the densitity function $f(x|\lambda)= \lambda e^{-\lambda x}$ with $\lambda > 0$ and observations $x_i \geq 0$.
We have :
\begin{align*}
    f(x|\lambda) &= \lambda e^{-\lambda x}\\
                &= e^{-\lambda x +ln(\lambda)}
\end{align*}

and the likelihood function:

\begin{align*}
    L(\lambda)= \prod_{i=1}^n f(x_i | \lambda)
\end{align*}
Therefore the log-likelihood is giveb by:

\begin{align*}
    l(\lambda)  &= ln(\prod_{i=1}^n f(x_i | \lambda)) \\
                &= \sum_{i=1}^n ln(f(x_i| \lambda))\\
                &= \sum_{i=1}^n ln( e^{-\lambda x_i +ln(\lambda)})\\
                &= \sum_{i=1}^n -\lambda x_i +ln(\lambda)
\end{align*}
With the same argument as in (a) we derive:
\begin{align*}
\frac{\delta l(\lambda)}{\delta \lambda}     &= \sum_{i=1}^n x_i + \frac{1}{\lambda} \\
                                            &= \frac{n}{\lambda} - \sum_{i=1}^n x_i
\end{align*}

Then we have :

\begin{align*}
    0  = \frac{\delta l(\lambda)}{\delta \lambda}    & \leftrightarrow \lambda = \frac{n}{\sum_{i=1}^n x_i}
\end{align*}
We also have 

\begin{align*}
    \frac{\delta^2 l(\lambda)}{\delta^2 \lambda} = -\frac{1}{\lambda^2}<0
\end{align*}

therefore we chose $\lambda_0 =  \frac{n}{\sum_{i=1}^n x_i}$ as our estimator.
\section*{Exercise 2}
\textbf{Given:} From our observation of the experiment and our prior knowledge we have :
\begin{align*}
    &p(\theta) = Dir(\theta \mid \alpha) \\
    &p( \mathcal{D} \mid \theta) = Mu(x \mid n,\theta)
\end{align*}
\textbf{Show} 
\begin{align*}
    p(\theta \mid \mathcal{D}) = Dir(\theta \mid \alpha+x)
\end{align*}

\begin{proof}
    Since the normalizing factor of a Distribution is dependent on the other factor of the distribution we can show the statement by showing :
    \begin{align*}
        p(\theta \mid \mathcal{D}) 	\propto Dir(\theta \mid \alpha + x)
    \end{align*}
From the lecture we know:
\begin{align*}
    p(\theta \mid \mathcal{D})  &= p(\theta) \cdot p( \mathcal{D} \mid \theta) \\
                                &= Dir(\theta \mid \alpha) \cdot Mu(x \mid n,\theta) \\
                                &= Z^{-1} \cdot \prod_{k=1}^K \theta_k^{a_k-1} \cdot Z^{-2} \cdot \prod_{j=1}^K \theta_j^{x_j}\\
                                &= Z^{-3}  \cdot \prod_{k=1}^K \theta_k^{a_k + x_k-1}\\
                                &\propto Dir(\theta \mid \alpha + x)
\end{align*}

With $Z^{-i}$  being normalizing factors. Therefore we have    
\[
     p(\theta \mid \mathcal{D}) = Dir(\theta \mid \alpha+x)
    \]
\end{proof}


\end{document}
