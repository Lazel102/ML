%This is my super simple Real Analysis Homework template

\documentclass{article}
\usepackage[english]{babel}
\usepackage[]{amsthm} %lets us use \begin{proof}
\usepackage[]{amssymb} %gives us the character \varnothing
\usepackage{mathtools}
\usepackage{amsmath}
\usepackage{lineno}
\usepackage[ansinew]{inputenc}
\newcommand{\R}{\mathbb{R}}
\newcommand{\uproman}[1]{\uppercase\expandafter{\romannumeral#1}}
\title{Solutions Sheet}
\author{Nina Fischer and Yannick Zelle}
\date\today
%This information doesn't actually show up on your document unless you use the maketitle command below

\begin{document}
\maketitle %This command prints the title based on information entered above

%Section and subsection automatically number unless you put the asterisk next to them.


%Basically, you type whatever text you want and use the $ sign to enter "math mode".
%For fancy calligraphy letters, use \mathcal{}
%Special characters are their own commands

\section*{Exercise 1}
\begin{itemize}
    \item [\textbf{(a)}] We have the random variable $X \sim \mathcal{U}(c,d)$. Therefore we have the pdf :
 \begin{align*}
 p_x(X) =   \begin{cases}
    \frac{1}{c-d} &c \leq x \leq d \\
    0 &\text{Otherwise}
    \end{cases}
\end{align*}

We now create the new random variable $Y(X)=aX+b$. Therefore we can deduct the inverse function:

\begin{align*}
    X(Y) = \frac{y-b}{a}
\end{align*}

Furthermore we can deduct:

\begin{align*}
    \frac{\delta X(Y)}{\delta Y} = \frac{1}{a}
\end{align*}

Since Y(X) is an increasing monotonic function, because $Y^{\prime}(X)=a>0$ according to therorem 7.1, we have :
\begin{align*}
p_y(Y) = p_x(X(Y)) \frac{\delta X(Y)}{\delta Y} = \begin{cases}
    \frac{1}{(c-d)a} &c \leq \frac{y-b}{a} \leq d \\
    0 &\text{Otherwise}
    \end{cases}
\end{align*}

\item[\textbf{(b)}] With the same argumentation as in a we can deduct

\begin{align*}
    p_y(Y)= \frac{1}{a(\sqrt{2\pi \sigma^2})}exp(-\frac{(\frac{Y-b}{a}- \mu)^2}{2 \sigma^2})
\end{align*}  
\end{itemize}

\section*{Exercise 2}

We have $p(\omega) = \mathcal{N}(\omega \mid \omega_0, V_0)$ and $p(y \mid X, \omega) = \mathcal{N}(y \mid X\omega, \sum)$.

\begin{align*}
&p(\omega \mid X, y) \\
&= p(y\mid X,\omega) \\
 &= \mathcal{N}(y\mid X\omega, \sum) \cdot p(\omega) \\
 &= exp(-\frac{1}{2} (y - X\omega) ^{T} \sum \nolimits ^{-1}(y - X\omega)) \cdot \mathcal{N}(\omega \mid \omega_0, V_0) \\
 &= exp(-\frac{1}{2} (y - X\omega) ^{T} \sum \nolimits ^{-1}(y - X\omega)) \cdot exp(-\frac{1}{2}(\omega - \omega_0) ^{T} V_0 ^{-1}(\omega - \omega_0)) \\
 &= exp(-\frac{1}{2} (y - X\omega) ^{T} \sum \nolimits ^{-1}(y - X\omega) -\frac{1}{2}(\omega - \omega_0) ^{T} V_0 ^{-1}(\omega - \omega_0)) \\
  &= exp -\frac{1}{2} ((y - X\omega) ^{T} \sum \nolimits ^{-1}(y - X\omega) + (\omega - \omega_0) ^{T} V_0 ^{-1}(\omega - \omega_0)) \\
  &= exp -\frac{1}{2}(y^{T} \sum \nolimits ^{-1} y- (X\omega)^{T} \sum \nolimits ^{-1} y - y ^{T} \sum \nolimits ^{-1} X\omega + (X\omega^{T} \sum \nolimits ^{-1} X \omega \\
  &+ w^{T} V_0^{-1} \omega - w_0^{T}V_0^{-1}\omega - \omega^{T} V_0^{-1}\omega_0 + \omega_0^{T} V_0^{-1} \omega_0) \\
  &= exp -\frac{1}{2}(y^{T} \sum \nolimits^{-1} y + \omega_0^{T}V_0^{-1}\omega_0 - (2y^{T} \sum \nolimits ^{-1} X + 2 \omega_0^{T} V_0^{-1}) \omega \\
  &+ \omega^{T}(V_0^{-1} + X^{T} \sum \nolimits ^{-1} X) \omega) \\
  &\prec exp(y^{T} \sum \nolimits ^{-1} X + \omega_0^{T} V_0^{-1}) \omega - \frac{1}{2} \omega^{T}(V_0^{-1} + X^{T} \sum \nolimits ^{-1} X) \omega) \\
  &= exp(\eta^{T} \omega - \frac{1}{2} \omega \Lambda \omega) \\
  &= \mathcal{N}(\omega \mid \omega_n, V_n)
\end{align*}
For the expressions $\Lambda$ and $\eta$ we can reordering :

\begin{align*}
\Lambda = V_n^{-1} = V_0^{-1} + X^{T} \sum \nolimits ^{-1} X \Leftrightarrow V_n = (V_0^{-1} + X^{T} \sum \nolimits^{-1} X)^{-1} \\
\end{align*}
\begin{align*}
\eta^{T} = V_n^{-1} \omega_n \Leftrightarrow \omega_n = \frac{\eta^{T}}{V_n^{-1}} = \eta^{T} V_n = (y^{T} \sum \nolimits ^{-1} X + \omega_0^{T} V_0^{-1})^{T} V_n \\
\end{align*}

\end{document}
