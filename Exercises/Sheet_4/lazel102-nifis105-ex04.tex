%This is my super simple Real Analysis Homework template

\documentclass{article}
\usepackage[english]{babel}
\usepackage[]{amsthm} %lets us use \begin{proof}
\usepackage[]{amssymb} %gives us the character \varnothing
\usepackage{mathtools}
\usepackage{amsmath}
\usepackage{lineno}
\usepackage[ansinew]{inputenc}
\newcommand{\R}{\mathbb{R}}
\newcommand{\uproman}[1]{\uppercase\expandafter{\romannumeral#1}}
\title{Solutions Sheet}
\author{Nina Fischer and Yannick Zelle}
\date\today
%This information doesn't actually show up on your document unless you use the maketitle command below

\begin{document}
\maketitle %This command prints the title based on information entered above

%Section and subsection automatically number unless you put the asterisk next to them.


%Basically, you type whatever text you want and use the $ sign to enter "math mode".
%For fancy calligraphy letters, use \mathcal{}
%Special characters are their own commands

\section*{Exercise 1}
\begin{itemize}
    \item [\textbf{(a)}] We have the random variable $X \sim \mathcal{U}(c,d)$. Therefore we have the pdf :
 \begin{align*}
 p_x(X) =   \begin{cases}
    \frac{1}{c-d} &c \leq x \leq d \\
    0 &\text{Otherwise}
    \end{cases}
\end{align*}

We now create the new random variable $Y(X)=aX+b$. Therefore we can deduct the inverse function:

\begin{align*}
    X(Y) = \frac{y-b}{a}
\end{align*}

Furthermore we can deduct:

\begin{align*}
    \frac{\delta X(Y)}{\delta Y} = \frac{1}{a}
\end{align*}

Since Y(X) is an increasing monotonic function, because $Y^{\prime}(X)=a>0$ according to therorem 7.1 :
\begin{align*}
p_y(Y) = p_x(X(Y)) \frac{\delta X(Y)}{\delta Y} = \begin{cases}
    \frac{1}{(c-d)a} &c \leq \frac{y-b}{a} \leq d \\
    0 &\text{Otherwise}
    \end{cases}
\end{align*}

\item[\textbf{(b)}] Wit the same argumentation as in a we can deduct

\begin{align*}
    p_y(Y)= \frac{1}{a(\sqrt{2\pi \sigma^2})}exp(-\frac{(\frac{Y-b}{a}- \mu)^2}{2 \sigma^2})
\end{align*}  
\end{itemize}

\section*{Exercise 2}

\section*{Exercise 2}


\end{document}
