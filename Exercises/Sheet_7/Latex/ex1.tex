
\begin{itemize}
    
\item[\textbf{(a)}] \textbf{Question:} Given a matrix $A \in \R^{nxn}$ and one of its eigenvectors $v$, how do you obtain the corresponding eigenvalue $\lambda$.
\newline
\textbf{Solution:} By definition of the eigenvector we have

\[
Av=\lambda v    
\]
Since we know $A$ and $v$ we can just plug them in and solve for $\lambda$.

\item[\textbf{(b)}] Show that scaling $v$ by a constant $c$ yields another eigenvector with the same $\lambda$
\begin{proof}
    Let $A \in \R^{nxn}$ and $v$ eigenvector of $A$ with corresponding eigenvalue $\lambda$. So we have 
\[
Av=\lambda v    
\]
Let now $c \in \R$ so we have for the vector $cv$:
\begin{align*}
    Acv  &= (Av)c\\
        &=(\lambda v)c\\
        &=\lambda cv
\end{align*}
Thus $cv$ is another eigenvector of a with the same corresponding eigenvalue $\lambda$
\end{proof}   
\item[\textbf{(c)}]
For symmetric $A \in \R^{nxn}$ with distinct eigenvalues $\lambda_1,...,\lambda_k$ show that the corresponding eigenvectors $v_1,...v_n$ are orthogonal to each other.

\begin{proof}
    Let $A \in \R^{nxn}$ be symmetric and $v_a, v_b$ two of the  eigenvectors with corresponding distinct eigenvalues $\lambda_a, \lambda_b$. With the definition of eigenvectors and the symmetrie of A we have:
    \begin{align}
        v_a^TAv_b=\lambda_a v_a^Tv_b \\
        v_a^TAv_b=\lambda_b v_a^Tv_b
    \end{align}
If we subtract (2) from (1) we have :

\[
0= (\lambda_a -\lambda_b)   v_a^Tv_b
\]
\end{proof}
And since $\lambda_a$ and $\lambda_b$ are distinct we have $(\lambda_a -\lambda_b) \neq 0$. Thus we have $v_a^Tv_b=0$. So per definition $v_a$ and $v_b$ are orthogonal to each other.
\end{itemize}