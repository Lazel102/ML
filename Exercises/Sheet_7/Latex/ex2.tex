
\begin{itemize}
    \item[\textbf{(a)}]
    \begin{proof}
        $\frac{1}{n}1_n1_n^T$ will be a matrix of the same form as $I$ with all entries $a_{ij}=\frac{1}{n}$ so H will have entries on the diagonal $a_{ii}=\frac{n-1}{n}$ and all other entries $a_{ij}=-\frac{1}{n}$.
        Since for a symmetric matirx to obtain the transposed all elements are mirrored on the diagonal we have :

        \[
        H^T=H    
        \]
    \end{proof} 
    \item[\textbf{(c)}] 
    We will proof that $H1_n= 0$
    \begin{proof}
        \begin{align*}
            H1_n    &=(I_n-\frac{1}{n}1_n1_n^T)1_n
                    &= I_n1_n -\frac{1}{n}1_n1_n^T1_n
                    &=1_n-\frac{1}{n}1_nn
                    &= 0
        \end{align*}
    \end{proof} 
    \item[\textbf{(b)}]
    We will show that H is idempotent.
    \begin{proof}
        \begin{align*}
            HH  &=H (I_n-\frac{1}{n}1_n1_n^T)\\
                &=HI_n -Hfrac{1}{n}1_n1_n^T\\
                &=H -\frac{1}{n}H1_n1_n^T\\
                &=H - \frac{1}{n}\cdot0\cdot 1_n^T\\
                &= H
        \end{align*}
        Thus H is idempotent.
    \end{proof} 
    \item[\textbf{(d)}] The other eigenvalue of $H$ is 1.
    \item[\textbf{(e}]We will show that $\frac{1}{n}XG1_n=0_d$ .
    \begin{proof}
        Let $X \in \R^{dxn}$ then :
        \begin{align*}
            \frac{1}{n}XH1_n &= \frac{1}{n}X0_n\\
                            &=\frac{1}{N}0_D\\
                            &= 0_D
        \end{align*}
    \end{proof} 
\end{itemize}