%This is my super simple Real Analysis Homework template

\documentclass{article}
\usepackage[ngerman]{babel}
\usepackage[]{amsthm} %lets us use \begin{proof}
\usepackage[]{amssymb} %gives us the character \varnothing
\usepackage{mathtools}
\usepackage{lineno}
\usepackage[ansinew]{inputenc}
\newcommand{\R}{\mathbb{R}}
\newcommand{\uproman}[1]{\uppercase\expandafter{\romannumeral#1}}
\title{L�sungen Kapitel 1}
\author{Yannick Zelle}
\date\today
%This information doesn't actually show up on your document unless you use the maketitle command below

\begin{document}
\maketitle %This command prints the title based on information entered above

%Section and subsection automatically number unless you put the asterisk next to them.


%Basically, you type whatever text you want and use the $ sign to enter "math mode".
%For fancy calligraphy letters, use \mathcal{}
%Special characters are their own commands

\section*{Exercise 1}
 First, we will assign values to $p(M), p(m), p(e), P(n\mid M), P(n\mid m), P(n\mid e)$, so that $p(M) + p(m) + p(e) = 1$:
 
 \[ p(M) := 0.01 \]
 \[ p(m) := 0.1 \]
 \[ p(e) := 0.89 \]
 
 \[ p(n\mid M)=0.9 \]
 \[ p(n\mid m)=0.08 \]
 \[ p(n\mid e)=0.02 \]
 
 The next step to apply Bayes' rule is now to calculate the probability of a noise:
 
 \[p(n)= p(M) \cdot p(n\mid M) + p(m) \cdot p(n\mid m) + p(e) \cdot p(n\mid e)\]
 
 \[ = 0.01 \cdot 0.9 + 0.1 \cdot 0.08 + 0.89 \cdot 0.02\]
 
 \[ = 0.0348\]
 
 We can now calculate the probability of a monster given noise by applying Bayes' rule:
 
 \[ p(M \mid n) =\frac{p(n \mid M) \cdot p(M)}{p(n)}
 \]
 \[=\frac{0.9 \cdot 0.01}{0.034} \]
 \[\approx 0.26\]



\section*{L�sung Aufgabe 1}
Das obenbeschrieben vorgehen l�sst sich als das aufeinanderfolgende durchf�hren von 2 Zufallsexperimenten verstehen. Wobei :

\begin{fleqn}%[1cm]
	\[
\Omega_1 = \{R_1, R_2, R_3, S_1, S_2, S_3, S_4 \} \]
\[
\Omega_2 = \{R_1, R_2, W_1, W_2, S_1, S_2, S_3 \}
\]
\end{fleqn}


Die Ergebnismenge des gesamten Experimentes l�sst sich dann beschrieben als :

\begin{fleqn}%[1cm]
	\[
\Omega= \{(\omega_1, \omega_2) \mid \omega_1 \in \Omega_1, \omega_2 \in \Omega_2 \}
\]
\end{fleqn}
 Wir bemerken dass, $card(\Omega) = card(\Omega_1) \cdot card(\Omega_2)=7 \cdot 7 =49$
 Betrachten wir das gefragte Ereignis $(\omega_i, \omega_j) \in A$. 
 Hierzu ben�tigen wir alle F�lle bei denen $\omega_i$ und $\omega_j$ die gleiche Farbe haben. Es gibt 3 rote Kugeln in $\Omega_1$ und 2 in $Omega_2$. Demnach gibt es $3 \cdot 2= 6 $ M�glichkeiten 2 rote Kugeln zu ziehen. F�r Schwarz folgen nach der selben Argumentation $4 \cdot 3 = 12$ M�glichkeiten. Demnach gilt $card(A)= 6+12=18$. Also ist die gesuchte Wahrscheinlichkeit:
 
\[
p((\omega_i,\omega_j))= \frac{card(A)}{card(\Omega)}= \frac{18}{49}
\]

\section{Aufgabe 2}

Ein W�rfel wird 7 mal geworfen. Wie hoch ist die Wahrscheinlichkeit, dass jeder der Zahlen 1-6 einmal unter den Wurfergebnissen vorkommt.

\section{L�sung Aufgabe 2}

Der Einfachheit halber k�nnen wir das durchgef�hrte Experiment, als eine Stichprobe ohne Reihenfolge mit R�cklegen verstehen. Wir w�rden also den Wurf $(1,1,2,3,4,5,6)$, genauso aufschreiben wie den Wurf $(6,6,5,4,3,2,1)$, n�mlich als $\{1,1,2,3,4,5,6 \}$.
Demnach ist unser Stichpropbenraum $\Omega_{\uproman{4}}$ mit:

\[
card(\Omega_{\uproman{4}})= \binom{N+n-1}{n}=\binom{12}{6}=\frac{12!}{6!\cdot 6!}=924
\]


Weiter �berlegen wir uns, dass in unserem Ergebnisraum $A$ jeweils ein Element doppelt vorkommen kann. Also ist

\[
card(A)=6
\]

Insgesamt gilt dann:

\[
p(A)=\frac{card(A)}{card(\Omega_{\uproman{4}})}=\frac{1}{154}
\]
\section{Aufgabe 3}
Unter 32 Karten befinden sich 4 Asse. Die Karten werden gemischt und nacheinander aufgedeckt.
Wie gro� ist die Wahrscheinlichkeit, dass die neunte aufgedeckte Karte das zweite aufgedeckte
Ass ist?

\section{L�sung Aufgabe 3}
Es handelt sich bei diesem Experiment um eine Stichprobe in Reihenfolge ohne Zur�clegen mit $N = 32$ und $n = 9$. Daher ist :

\[
card(\Omega) = (N)_n = N(N-1)\cdot ... \cdot (N-n+1)= 32 \cdot 31 \cdot ... \cdot 22
\]

F�r die gr��e des uns interessierendes Ereignis $A$ l�sst sich Argumentieren, dass wir 4 Asse haben, die jeweils an 8 Positionen seien k�nnen. An der 9. Position bleiben noch 3 Asse �brig. �brig bleiben 7 Positionen, die sich als Permutation der 28 Karten verstehen lassen, die keine Asse sind. Insgesamt ergibt sich also:

\[
card(A) = 4*3*8*(28)_7
\]
 Demnach k�nnen wir die Wahrscheinlichkeit berechnen mit:
 
\[
p(A)=\frac{card(A)}{card\Omega}=\frac{4*3*8*(28)_7}{(32)_9
}
\] 

\section{Aufgabe 4}

Die Ecken eines Wiirfels sind gleichm?Big schr?g abgeschliffen worden, so dass der Wiirfel auch
auf jeder dieser Ecken liegen bleiben kann. Allerdings ist die Wahrscheinlichkeit jeder Ecke nur
1/4 so groB wie die jeder Seite. Wie groB ist die Wahrscheinlichkeit einer Sechs?

\section{L�sung Aufgabe 4}

Da ein W�rfel 6 Seiten und 8 Ecken hat Enth�lt $\Omega$ Insgesamt 14 Elemente. Wit bezeichnen in der Folge Ecken mit $e$ und Seiten mit $s$. Da alle Ecken und alle Seiten jeweils die gleiche Wahrscheinlichkeit haben und $P(\Omega)=1$ gilt:
\[
6 \cdot p(s)+ 8 \cdot p(e) = 1 \\

\]
Zus�tzlich wissen wir aus der Aufgabenstellung, dass:

\[
4\cdot p(e)=p(s) \newline
\]
\[
\Leftrightarrow \frac{1}{4}\cdot p(s)=p(e)
\]

Setzen wir dies oben ein ergibt sich :

\[
p(s)=\frac{1}{8}=p(6)
\]



\end{document}
